\documentclass[10pt,a4paper]{article}
\usepackage[T1]{fontenc}
\usepackage[utf8]{inputenc}
\usepackage{lmodern}

% Layout and styling
\usepackage[margin=2cm]{geometry}
\usepackage{titlesec}
\usepackage{enumitem}
\usepackage{xcolor}
\usepackage[hidelinks]{hyperref}
\usepackage{microtype}

% Spacing
\setlength{\parindent}{0pt}
\setlength{\parskip}{0.15em}
\setlist[itemize]{itemsep=0.12em, topsep=0.12em, left=1.2em}

% Section styling
\newcommand{\sectionseparator}{\vspace{-0.3em}\noindent\color{black!20}\rule{\linewidth}{0.5pt}\vspace{0.2em}}
\titleformat{\section}{\large\bfseries}{}{0pt}{}[\sectionseparator]
\titlespacing*{\section}{0pt}{0.6em}{0.2em}

% Entry layout macro
\newcommand{\entry}[3]{\noindent\textbf{#1} — #2\hfill #3\\}

% Tech stack macro
\newcommand{\techstack}[1]{\textcolor{black!70}{\small\textit{#1}}}

% Personal details
\newcommand{\Name}{Karolis Strazdas}
\newcommand{\Email}{karolis.strazdas@ktu.edu}
\newcommand{\Website}{https://karolis-strazdas.lt}
\newcommand{\GitHub}{https://github.com/Eoic}
\newcommand{\LinkedIn}{https://www.linkedin.com/in/karolis-strazdas/}

\begin{document}
\pagestyle{plain}  % Add page numbers for multi-page academic CV

\begin{center}
{\Huge\bfseries \Name}\\[0.8em]
{\small \href{mailto:\Email}{\Email}\quad\textbullet\quad \href{\Website}{karolis-strazdas.lt}\quad\textbullet\quad \href{\GitHub}{github.com/Eoic}\quad\textbullet\quad \href{\LinkedIn}{linkedin.com/in/karolis-strazdas}}
\end{center}

\vspace{-0.5em}

% Remove the small font group for better readability on 2-page format

\section*{Research interests}
Machine learning, multivariate time series analysis, sensor-based defect detection, computer vision, and software engineering.

\section*{Education}
\entry{Kaunas University of Technology}{Master of Artificial Intelligence in Computer Science}{2025 -- 2027\footnote{Expected completion.}}
\vspace{-0.8em}
\begin{itemize}[leftmargin=*, topsep=0em]
  \item Key coursework includes modern AI methods and techniques, machine learning, deep learning, data analysis, cloud computing, project management and computer vision.
  \item Research interests focus on modelling of sensor data, time-series forecasting, and applications of AI in manufacturing and engineering.
\end{itemize}

\entry{Kaunas University of Technology}{Bachelor of Software Engineering}{2016 -- 2020}
\vspace{-0.8em}
\begin{itemize}[leftmargin=*, topsep=0em]
  \item Completed a comprehensive curriculum in software engineering, covering programming languages, data structures, algorithms and system design.
  \item Bachelor's thesis project: \emph{Debt-recovery management system} - designed and implemented a web platform for automating case intake, repayment plans and tiered reminder schedules; built backend and database schema with Django and PostgreSQL; developed document and CSV uploading and processing features; deployed the system behind Apache.
\end{itemize}

\section*{Research and professional experience}
\entry{Indeform Ltd.}{Software Engineer / Research Associate}{Jul 2020 -- Present}
\vspace{-0.8em}
\begin{itemize}[leftmargin=*, topsep=0em]
  \item Developed machine learning pipelines for wear prediction and defect detection in multivariate time series data from stamping presses (piezoelectric, force, vibration, acoustic emission and displacement sensors). Achieved F1 $\approx$ 0.94, MAE $\approx$ 0.09mm and $R^2$ $\approx$ 0.96; authored a literature review and project reports summarising data processing pipelines, model performance and real-world applicability.
  \item Implemented and ported 3D mesh editing features (polyline slicing, deformations) for an orthopaedic modelling platform; translated Rhino scripts to WebGL, enabling precise mesh modifications within a browser (Django, MySQL, Three.js).
  \item Built a company-wide internal web platform to manage projects, teams, onboarding and communication with a custom role-based access system, improving collaboration and documentation.
  \item Established continuous integration and deployment workflows using Docker, Nginx and Jenkins for both AWS EC2 and on-premises servers; configured load balancing and rollback strategies to minimise downtime.
  \item Prototyped interactive projection installations for museums using OpenCV, C++, Python, Godot and ImGui; implemented real-time ball tracking and parallelised camera frame buffering for robust input handling.
  \item Contributed to software quality assurance by authoring TestLink scenarios and adding Cypress integration tests for critical workflows.
\end{itemize}

\entry{Indeform Ltd.}{Software Engineer Intern}{Feb 2020 -- May 2020}
\vspace{-0.8em}
\begin{itemize}[leftmargin=*, topsep=0em]
  \item Participated in requirements analysis and domain modelling sessions for a debt recovery web application; helped define scope, workflows and success criteria.
  \item Designed and built an MVP debt-recovery platform featuring case intake, repayment plan creation, tiered email reminders and reporting; implemented document and CSV uploading, processing and management.
  \item Designed database schema and REST APIs using Django and PostgreSQL; added unit and integration tests; deployed behind Apache.
  \item This project served as the basis for the Bachelor's thesis.
\end{itemize}

\section*{Academic and open-source projects}
\begin{itemize}[leftmargin=*, topsep=0em]
  \item \textbf{TimeLab} -- Local-first tool for labelling multivariate time-series data used in research on sensor-based defect detection. \techstack{TypeScript, SCSS} | \url{https://github.com/Eoic/TimeLab}
  \item \textbf{Papyrus} -- Cross-platform e-book management system for organising digital libraries and synchronising across devices. \techstack{Python, Dart, Flutter, FastAPI, PostgreSQL} | \url{https://github.com/Eoic/Papyrus}
  \item \textbf{ASCIIGround} -- TypeScript library for rendering animated ASCII backgrounds. \techstack{TypeScript} | \url{https://github.com/Eoic/ASCIIGround}
  \item \textbf{NetBots} -- Real-time multiplayer robot programming game exposing learners to algorithms and distributed systems. \techstack{TypeScript, Python, Rust, MongoDB, Pixi.js} | \url{https://github.com/Eoic/NetBots}
  \item \textbf{Canopy} -- Real-time spatial conversation platform providing playful and immersive communication space. \techstack{Python, TypeScript, Pixi.js} | \url{https://github.com/Eoic/Canopy}
\end{itemize}

\section*{Technical skills}
\textbf{Programming languages}: Python, R, JavaScript/TypeScript, Elixir, Rust.\\
\textbf{Machine learning and data analysis}: scikit-learn, TensorFlow, PyTorch, OpenCV.\\
\textbf{DevOps and infrastructure}: Docker, Docker Compose, Nginx, Apache, Jenkins, AWS.\\
\textbf{Frameworks and tools}: Django, FastAPI, Three.js, Godot, ImGui.

\section*{Languages}
Lithuanian — native; English — professional working proficiency.

\section*{Interests}
Cycling, running, reading, personal programming projects, and continuous learning in technology and science.

\end{document}
